\section{Introduction}
Query optimization is a long-established topic in database management systems. For a given declarative query submitted by a user, there are many possible execution plans, each of which describes a correct way to calculate the result. The plans for a query can vary in cost by orders of magnitude, so a query optimizer is vital to choose an efficient plan among the possibilities. Most database management systems include an optimizer that is cost-based: it considers a variety of plans, estimates the cost of each plan from statistics, knowledge of the index structures, etc., and then chooses to execute the plan with lowest estimated cost among those it considered. 

MongoDB~\cite{mongodb_2019} is a popular document-oriented DBMS with a very different approach to query optimization which is not based on estimating the costs of queries before they are run. Instead, MongoDB  runs many execution plans in a round-robin ``race'', allowing each to do a small amount of work at a time. After a point, it considers the progress of each plan and calculates a score based on the number of results produced for the work done to that point. The plan with the highest score wins the race, and it alone runs to completion as the chosen plan for this query. We call the MongoDB technique "first past the post" (\approachName) query optimization.  
%As far as we know, no previous research has evaluated the effectiveness of the \approachName approach.

The central aim of our work is to evaluate and understand the implementation of MongoDB's \approachName optimization. In this paper, we explain the query optimization approach of MongoDB in Section~\ref{sec:background}. We describe our innovative approach to evaluation of query optimization %(and how we visualize the results)
in Section~\ref{sec:methodology}.
The results of our empirical study of MongoDB are presented in Section~\ref{sec:evaluation}. We find that \approachName in MongoDB has a preference bias and systematically avoids collection scans even when they are the best plan. We explore the reasons for this in Section~\ref{sec:rootcauseanalysis} and then propose an improvement and evaluate its effectiveness.

This paper makes the following contributions:
\begin{enumerate}
    \item We describe in detail how the \approachName query optimizer in MongoDB chooses query plans.
    \item We propose an innovative way to evaluate and visualize the impact on query performance of an optimizer's choices. By using this approach, we identify places where the MongoDB query optimizer chooses suboptimal query plans.
    Our approach could form the basis of an automated regression testing tool to verify that the query planner in MongoDB improves over time.
    \item We identify causes of the preference bias of \approachName, in which index scans are systematically chosen even when a collection scan would run faster.
\end{enumerate}
